\documentclass{article}

\usepackage{listings}
\lstnewenvironment{code}{}{}
\usepackage{hyperref}
\usepackage{color}
\usepackage{graphicx}

\definecolor{KBlue}{rgb}{0.0,0.2196,0.5098}    % 0, 56, 130
\definecolor{listinggray}{gray}{0.95}

\newcommand{\footnoteremember}[2]{
  \footnote{#2}
  \newcounter{#1}
  \setcounter{#1}{\value{footnote}}
}
\newcommand{\footnoterecall}[1]{\footnotemark[\value{#1}]}

\lstdefinelanguage[improved]{Haskell}
   % To separate out word keywords from symbol keywords for different formatting,
   % we define the word keywords as emph items (use emphstyle):
  {classoffset=0,
   %If we don't specify at least one "non-other" keyword, listings doesn't work, hence:
   morekeywords={hduisahfiuabfyasbyoasvbfuyvosf},
   otherkeywords={::,=,==,->,=>,>>,>>=,>>*,$,++,<-,<|>,<->,<||>,</>,\\,.,__,<&>},
   classoffset=1,
   morekeywords={data,forall,type,module,newtype,let,in,do,where,if,then,else,qualified,as,import},
   % For some (unknown) reason, setting classoffset = 0 again after this line
   % breaks the highlighting.
   morecomment=[l]{--},
%   morestring=[b]',
   morestring=[b]",
  }
%$
\lstset{
	language={[improved]Haskell},
	columns=flexible,
        backgroundcolor=\color{listinggray},
        frameround=tttt,
        frame=trbl,
        framerule=0.4pt,
	basicstyle=\small\sffamily,
	emphstyle=\bfseries,
        keywordstyle=[1]{\color{KBlue}\bfseries},
        keywordstyle=[0]{\color{KBlue}\bfseries\ttfamily},
	identifierstyle=,
	commentstyle=,
	stringstyle=\ttfamily,
	showstringspaces=false}
\thicklines

\title{Alloy tutorial}
\author{Neil C. C. Brown}

\begin{document}

\maketitle

\section*{Introduction}

This document is a tutorial for the Alloy generics library.  Alloy is similar
to other generics libraries, such as Scrap Your Boilerplate (SYB), Uniplate,
EMGM and all the rest.  Alloy tends to be quite fast (see our paper for
benchmarks) because it avoids traversing parts of the data structure that it
does not need to.

This is accomplished by generating type-class instances based on the
can-contain relation between types.  The current set of operations (opset) is
trimmed dynamically to remove types that can no longer be contained in the
data item being traversed.  For more details, see the draft paper.
%TODO

%\newpage
%\tableofcontents

\newpage
\section{Paradise Benchmark}
\input{CompanyDatatypes.lhs}
\newpage
\subsection{The Basics}
\input{GenTypes.lhs}
\input{Paradise.lhs}
\newpage
\subsection{Multiple Target Types and Controlled Descent}
\input{Selective.lhs}
\subsection{Type-Class Constraints}
\input{Constraints.lhs}

\newpage
\subsection{Effects}
\input{Effects.lhs}
\newpage
\input{EffectsSelective.lhs}

\newpage
\subsection{Queries}
\input{Query.lhs}

\newpage
\subsection{Routes}
\input{Routes.lhs}

\newpage
\subsection{Maps and Sets}
\input{MapSet.lhs}
\input{GenTypesMapSet.lhs}
\input{MapSetExample.lhs}

%\newpage
%\section{Frequently Asked Questions}

\end{document}
